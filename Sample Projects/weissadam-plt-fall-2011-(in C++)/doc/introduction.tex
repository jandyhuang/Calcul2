\chapter{Introduction}
The \lang language was created to leverage the clear and concise syntax of set formalisms in mathematics.  With \lang, users can generate sets using a variety of logic and combinatorial techniques that would require significantly more lines of code in other imperative languages such as Java, C or C++.  \lang was designed with the following goals in mind:
\begin{itemize}
\item\textbf{Set Theory Abstraction. }\lang provides a framework for handling data which many mathematicians and scientists will find familiar and easy to use.  By maintaining a sufficient level of abstraction, \lang can minimize the time needed to go from concept to concrete, working code.  

\item \textbf{Minimal Code Generation. } \lang greatly reduces the amount of code needed to perform routine tasks.  The user will find that nested loops can be virtually eliminated from code, making program files lighter and debugging easier.

\item \textbf{High Level of Readability. } \lang aims to be a clear and concise programming language with intuitive commands and syntax which mirror the mathematical underpinnings of set theory.
\end{itemize}