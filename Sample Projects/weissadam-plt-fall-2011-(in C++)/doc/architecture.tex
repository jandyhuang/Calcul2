\chapter{Compiler Architecture}
The compiler consists of 5 principal components: Scanner, Parser, Validator, Code Generator and Set Class.  The Scanner receives input in the form of plain text files with \verb|.su| suffix and the compiler generates valid cpp code.  

The Code Generator uses the concrete parse tree generated from the Parser to output valid C++ code.  A polymorphic class hierarchy was implemented in C++ (Set Class component) to implement complicated Tuple and Set types.

The following lays out who worked on the various component implementations:

\begin{center}
\begin{tabular}{lc}
\toprule
\textbf{Component} &\textbf{Team Members}\\\midrule
Scanner	&Adam\\
Parser	&Adam, Ian\\
Validation &Bill\\
Code Generator &Andrew, Bill, Ian\\
Set Class &Bill, Andrew\\
Regression Suite &Ian, Bill, Andrew\\\bottomrule
\end{tabular}
\end{center}

\tikzstyle{n1} = [minimum height = 3em, minimum width = 10em, text centered, fill=blue!40, drop shadow, draw = blue, rounded corners]
\tikzstyle{n2} = [minimum height = 3em, minimum width = 10em, text centered, fill=orange!40, drop shadow, draw = orange, rounded corners]
\tikzstyle{n3} = [minimum height = 3em, minimum width = 10em, text centered, fill=black!40, drop shadow, draw = black, rounded corners]
\tikzstyle{n4} = [minimum height = 3em, minimum width = 10em, text centered, fill=green!40, drop shadow, draw = green, rounded corners]
\tikzstyle{toarrow} = [->, >= open triangle 90, thick]

\begin{figure}[h]
\label{arch}
\begin{tikzpicture}
	\node (input) [n3] at (5,2.5) {\Large infile.su};
    \node (s) [n1] at (5,0) {\Large Scanner};
    \node (p) [n1] at (5,-2.5) {\Large Parser};
    \node (v) [n2] at (0,-2.5) {\Large Validator};
	\node (cw) [n1] at (5,-5) {\Large Code Generator};
	\node (cpp) [n2] at (0,-5) {\Large Set Class };
    \node (out) [n4] at (5,-7.5) { \Large outfile.cpp };
	\node (test) [n2] at (10,-5) { \Large Regression Suite };
	
	\draw [toarrow] (input) -- (s);    
    \draw [toarrow] (s) -- (p);
    \draw [toarrow] (p) -- (v);
    \draw [toarrow] (p) -- (cw);
    \draw [toarrow] (v) -- (cw);
    \draw [toarrow] (cpp) -- (cw);
    \draw [toarrow] (cw) -- (out);
    \draw [toarrow] (test) -- (out);
    \draw [toarrow] (test) -- (cw);
\end{tikzpicture}
\caption{Overview of Setup Compiler}
\end{figure}
