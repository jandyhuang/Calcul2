\chapter{Setup Tutorial}
Each program you write in \lang should be placed in a plain text file.  By convention, the \verb|.su| suffix is used to denote programs written in valid \lang code.  Let's write a simple program to get the hang for how \lang works.

\section{A Simple Scheduling Program}
\subsection{Writing the Code}
Create a text file entitled \verb|schedule.su| and open it in your text editor.  In this file, place the following text:

\begin{code}
/* Our First Program */
function main[] returns int {
   set days = {"M","T","W","Th","F"};
   set hrs = {1...24};
   set week = Days cross hrs;
   return 0;   
   }
\end{code}

We use the keyword \verb|set| to indicate that we are defining a variable name for a set
\subsection{Compiling a Program}

\begin{code}

-$ ./cWriter.native < hello.su

\end{code}

\section{Sets and Tuples}
\section{Using Set-Builder Notation}
