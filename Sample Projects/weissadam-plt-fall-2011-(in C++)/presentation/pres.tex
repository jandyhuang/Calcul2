\documentclass{beamer}
\usepackage{beamerthemeshadow}
\usepackage{tikz}
\usetikzlibrary{shadows,positioning,arrows,fit,petri,decorations.pathmorphing,backgrounds}

\title{\textsc{The Setup Programming Language}}
\author{I. Erb - A. Ingraham - B. Miller - A. Weis}

\begin{document}
\frame{\titlepage}

\section*{An Overview of Setup}
\begin{frame}
	\frametitle{Overview and Motivation}
	\framesubtitle{Setup is a language for manipulating sets and tuples}
	\begin{block}{Overview}
	\begin{itemize}
	\item Uses \emph{C-style} syntax
	\item Additional \textit{tuple} and \textit{set} objects
	\item Elementary operations on sets are supported (e.g., union, intersection, set-difference)
	\end{itemize}
	\end{block}

	\begin{block}{Motivation}
	\begin{itemize}
	\item The language of set theory is widely-used in mathematics and science
	\item Programs can be written quickly without verbose code
	\item Natural and intuitive way to think about large data sets
	\end{itemize}
	\end{block}	
\end{frame}

\begin{frame}
	\frametitle{Objectives}
	\framesubtitle{In creating Setup, we had three objectives in mind}
	\begin{block}{Objectives}	
	\begin{enumerate}
		\item Create, modify and manipulate sets and tuples in a natural and intuitive way
		\item Minimal code generation
		\item Allow for both ordered and unordered collections of data
	\end{enumerate}
	\end{block}
\end{frame}

\section*{How to Use Setup}
\begin{frame}
\frametitle{The Basics}

\begin{block}{Basic Elements of our Language}
\begin{center}
\begin{tabular}{ccc}
\textbf{Primatives} &\textbf{Objects} &\textbf{Set Operators}\\\hline
\texttt{int} &\texttt{set} &\texttt{union}\\
\texttt{string} &\texttt{tuple} &\texttt{intersect}\\
\texttt{double} & &\texttt{cross}\\
\texttt{bool} & &\texttt{minus}\\
 & &\texttt{\#}
\end{tabular}
\end{center}
\end{block}

\begin{itemize}
\item The usual arithmetic operations \texttt{+,-,*,/} are provided
\item The \texttt{+} operator is overloaded for string concatenation
\end{itemize}
\end{frame}

\begin{frame}
\frametitle{Sets and Tuples}
\begin{block}{Tuples}
\texttt{((1,"a"),2) //type: ((int,str),int)}\\
\texttt{((1,2),3) //type: ((int,int),int)}
\end{block}

\begin{block}{Set Initialization}
\texttt{set A = {4, 5, 6};}\\
\texttt{set B = {1 ... 6};}\\
\texttt{set C = { (x,y) | x in A, y in B };}
\end{block}

\end{frame}

\begin{frame}
\frametitle{Example Functions}
\begin{block}{Functions}
\texttt{function main[] returns int \{/* ... */\}}\\
\texttt{function symDiff[set A, set B] returns set \{}\\
\texttt{\qquad\qquad return (A union B) minus (A intersect B);}
\texttt{\}}

\end{block}
\begin{itemize}
\item Execution begins with \texttt{main}
\item Functions must be defined before they are called
\item Function definition nesting is not permitted
\end{itemize}
\end{frame}

\section*{Implementation}
\tikzstyle{n1} = [text centered, fill=blue!40, drop shadow, draw = blue, rounded corners]
\tikzstyle{n2} = [text centered, fill=orange!40, drop shadow, draw = orange, rounded corners]
\tikzstyle{n3} = [text centered, fill=black!40, drop shadow, draw = black, rounded corners]
\tikzstyle{n4} = [text centered, fill=green!40, drop shadow, draw = green, rounded corners]
\tikzstyle{toarrow} = [->, >= open triangle 90, thick]
\begin{frame}
\begin{figure}[h]
\label{arch}
\begin{tikzpicture}
	\node (input) [n3] at (0,-5) {infile.su};
    \node (s) [n1] at (2,-5) {Scanner};
    \node (p) [n1] at (2,-3) {Parser};
    \node (v) [n2] at (5,-3) {Validator};
	\node (cw) [n1] at (5,-5) {cWriter};
	\node (cpp) [n2] at (2,-6) { SetupBase };
    \node (out) [n4] at (5,-6) {outfile.cpp };
	
	\draw [toarrow] (input) -- (s);    
    \draw [toarrow] (s) -- (p);
    \draw [toarrow] (p) -- (v);
    \draw [toarrow] (p) -- (cw);
    \draw [toarrow] (v) -- (cw);
    \draw [toarrow] (cpp) -- (out);
    \draw [toarrow] (cw) -- (out);
\end{tikzpicture}
\caption{Overview of Setup Compiler}
\end{figure}
\end{frame}

\begin{frame}
\frametitle{Implementation}
\begin{block}{Implementation}
\begin{itemize}
\item Compiler can verify most type assignments by looking in a single frame of function context 
\item Setup supports a compact syntax we call the Set Builder expressions
\item Set Builder has a lambda style syntax 
\item Variables are defined by the result of an expression evaluation
\end{itemize}
\end{block}
\end{frame}

\section*{Lessons}
\begin{frame}
\frametitle{Lessons Learned}
\begin{block}{Lessons Learned}
\begin{itemize}
	\item Do not underestimate the task of building a compiler
	\item Just because the road is at first downhill, does not mean it does not lead to a cliff
\end{itemize}
\end{block}
\end{frame}
\end{document}